We have used optical observations of resolved stars from the Panchromatic
Hubble Andromeda Treasury (PHAT) to measure the recent ($< 500\myr$) star
formation histories (SFHs) of 33 \fuv{}-bright regions in M31. The region areas
ranged from $\sim 10^4$ to $10^6\pc^2$, which allowed us to test the
reliability of \fuv{} flux as a tracer of recent star formation on sub-kpc
scales. The star formation rates (SFRs) derived from the extinction-corrected
observed \fuv{} fluxes were, on average, consistent with the 100-Myr mean SFRs
of the SFHs to within the $1\sigma$ scatter. Overall, the scatter was larger
than the uncertainties in the SFRs and particularly evident among the smallest
regions. The scatter was consistent with an even combination of discrete
sampling of the initial mass function and high variability in the SFHs. This
result demonstrates the importance of satisfying both the full-IMF and the
constant-SFR assumptions for obtaining precise SFR estimates from \fuv{} flux.
Assuming a robust \fuv{} extinction correction, we estimate that a factor of
2.5 uncertainty can be expected in \fuv{}-based SFRs for regions smaller than
$10^5\pc^2$, or a few hundred pc. We also examined ages and masses derived from
UV flux under the common assumption that the regions are simple stellar
populations (SSPs). The SFHs showed that most of the regions are not SSPs, and
the age and mass estimates were correspondingly discrepant from the SFHs. For
those regions with SSP-like SFHs, we found mean discrepancies of $10\myr$ in
age and a factor of 3 to 4 in mass. It was not possible to distinguish the
SSP-like regions from the others based on integrated \fuv{} flux.

Starting from star formation histories based on color magnitude diagrams, we
have used stellar population synthesis to create maps of synthetic far- and
near-ultraviolet (\fuv{} and \nuv{}) flux at sub-\kpc{} resolution for the
northeast quadrant of M31. The synthetic maps reproduce all of the main
morphological features found in corresponding maps of observed \fuv{} and
\nuv{} flux, including rings and large star-forming complexes. Comparing the
synthetic and observed flux maps pixel-by-pixel, we found the fluxes to be
consistent, on average, to within factors of 1.7 and 1.4 for \fuv{} and \nuv{},
respectively. We used the observed fluxes and standard flux calibrations to
derive star formation rate (SFR) maps, which we compared with a map of the mean
SFRs over the last $100\myr$ of the SFHs. We determined a lower limit of $\sfr
\sim 10^{-5}\msun\yr^{-1}$ below which the commonly assumed linear relationship
between flux and SFR appears to break down. Above this limit, the flux-based
SFRs were consistent with the mean SFRs, on average, to within factors of 1.8
for \fuv{} and 1.7 for \nuv{}. We calculated a global SFR of $0.3\msun\yr^{-1}$
for the entire mean SFR map. The corresponding measurements from the flux-based
SFR maps were 30\% lower and 40\% higher for \fuv{} and \nuv{}, respectively.
The primary source of uncertainty in both the synthetic flux maps and the
flux-based SFR maps was most likely incomplete IMF sampling due to the small
pixel areas. Excepting the faintest areas of the galaxy, we did not identify
any trends for flux or SFR with environment.

Pull it together with a short paragraph.
