We have used optical observations of resolved stars from the Panchromatic
Hubble Andromeda Treasury (PHAT) to measure the recent ($< 500\myr$) star
formation histories (SFHs) of 33 \fuv{}-bright regions in M31. The region areas
ranged from $\sim 10^4$ to $10^6\pc^2$, which allowed us to test the
reliability of \fuv{} flux as a tracer of recent star formation on sub-kpc
scales. The star formation rates (SFRs) derived from the extinction-corrected
observed \fuv{} fluxes were, on average, consistent with the 100-Myr mean SFRs
of the SFHs to within the $1\sigma$ scatter. Overall, the scatter was larger
than the uncertainties in the SFRs and particularly evident among the smallest
regions. The scatter was consistent with an even combination of discrete
sampling of the initial mass function and high variability in the SFHs. This
result demonstrates the importance of satisfying both the full-IMF and the
constant-SFR assumptions for obtaining precise SFR estimates from \fuv{} flux.
Assuming a robust \fuv{} extinction correction, we estimate that a factor of
2.5 uncertainty can be expected in \fuv{}-based SFRs for regions smaller than
$10^5\pc^2$, or a few hundred pc. We also examined ages and masses derived from
UV flux under the common assumption that the regions are simple stellar
populations (SSPs). The SFHs showed that most of the regions are not SSPs, and
the age and mass estimates were correspondingly discrepant from the SFHs. For
those regions with SSP-like SFHs, we found mean discrepancies of $10\myr$ in
age and a factor of 3 to 4 in mass. It was not possible to distinguish the
SSP-like regions from the others based on integrated \fuv{} flux.

Starting from SFHs derived from the full PHAT photometric dataset, we
have used stellar population synthesis to create maps of synthetic far- and
near-ultraviolet (\fuv{} and \nuv{}) flux at sub-kpc resolution for the
northeast quadrant of M31. The synthetic maps reproduced all of the main
morphological features found in corresponding maps of observed \fuv{} and
\nuv{} flux, including rings and large star-forming complexes. Comparing the
flux maps pixel-by-pixel, we found the median synthetic-to-observed flux ratios
to be $1.02 \;+\!0.74/\!-\!0.43$ in \fuv{} and $0.79 \;+\!0.35/\!-\!0.24$ in
\nuv{}. The synthetic fluxes were therefore consistent overall with the
observed fluxes in both filters. We used the observed fluxes and standard flux
calibrations to derive star formation rate (SFR) maps, which we compared with a
map of the mean SFRs over the last $100\myr$ of the star formation histories
(SFHs). We determined a lower limit of $\sfr \sim 10^{-5}\msun\yr^{-1}$ below
which the commonly assumed linear relationship between UV flux and SFR appears
to break down. Above this limit, we found the median ratios of the flux-based
SFRs to the mean SFRs to be $0.57 \;+\!0.47/\!-\!0.26$ in \fuv{} and $1.24
\;+\!0.88/\!-\!0.52$ in \nuv{}. Both the \fuv{} and \nuv{} flux-based SFRs were
therefore consistent overall with the mean SFRs derived from the SFHs.
Integrating over the entire mean SFR map, we found a global SFR of
$0.3\msun\yr^{-1}$. The corresponding measurements from the flux-based SFR maps
were factors of 0.74 (\fuv{}) and 1.45 (\nuv{}) of the global mean SFR value.
It is not yet understood why the SFR ratios in the global case are larger than
the median pixel-wise ratios. The primary source of uncertainty in both the
synthetic flux maps and the flux-based SFR maps was most likely incomplete IMF
sampling due to the small pixel areas. With the exception of the faintest areas
of the galaxy, we did not identify any trends for flux or SFR with environment.
