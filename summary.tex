\chapter{Summary}
\label{summary}

In this study, we have derived the recent ($< 500\myr$) SFHs of 33
UV-bright regions in M31 using optical HST observations from PHAT. The regions
were defined by \citetalias{Kang:2009} based on GALEX \fuv{} surface brightness and
have areas ranging from $8 \times 10^3$ to $1.5 \times 10^6\pc^2$. We
used the SFH code MATCH to fit the CMDs of the regions and measure their the
SFHs based on the resolved stars from the PHAT photometry. We modeled the
extinction in the regions using a foreground parameter and a differential
parameter, which were optimized for each region to find the best-fit SFH.

We used FSPS to model both the intrinsic and reddened \fuv{} and \nuv{} magnitudes of
the regions based on their SFHs. The differences between the modeled reddened
and the observed \fuv{} magnitudes, $\fuvsfh - \fuvobs$, followed a
normal distribution with $\mu=0.09$ and $\sigma=0.3$. On average, the
\fuvsfh{} values were consistent with the \fuvobs{}
values, confirming the reliability of the SFHs, our extinction model, and the
\citet{Cardelli:1989} extinction curve. We attribute the scatter in the flux
ratios to the assumption made by FSPS that the IMF is fully populated while the
actual distribution of stellar masses becomes more discrete as smaller regions
are considered.

The observed, extinction-corrected \fuv{} magnitudes were converted into SFRs,
\sfrfuv{}, using the \fuv{} flux calibration from \citet{Kennicutt:1998}
with updates by \citet{Hao:2011} and \citet{Murphy:2011}. We also derived the mean
SFRs for the last $100\myr$ of the SFHs, \sfroneh{}. The $\sfrfuv / \sfroneh$
ratios were log-normally distributed with $\mu=0.2$
and $\sigma=0.4$. Overall, the \sfrfuv{} values were consistent with
the \sfroneh{} values, though a small amount of the
offset was attributable to inconsistencies with the metallicity assumed by the
flux calibration.

The intrinsic modeled \fuv{} magnitudes were also converted into SFRs,
\sfrfuvz{}, which were free from biases due to extinction
corrections and IMF sampling. The log-normal for the
$\sfrfuvz / \sfroneh$ ratios had $\mu=0.1$
and $\sigma=0.3$, indicating that assuming a constant SFR (implicit in the flux
calibration) for regions with highly variable SFHs is an important source of
scatter. We conclude that the total scatter in the $\sfrfuv / \sfroneh$ ratio
is due to the assumptions of a full IMF and a
constant SFR in regions where discrete sampling of the IMF and high variability
in the SFHs are important. Combined, these effects result in a factor of 2.5
uncertainty in the \fuv{}-based SFRs. Although there is a significant lack of
regions in our sample with areas between $10^5$ and $10^6\pc^2$, we
estimate that discrete IMF sampling and SFH variability become important below
$10^5\pc^2$, or scales of a few hundred pc.

Ages and masses were derived for the regions by \citetalias{Kang:2009} from
observed $\fuv - \nuv$ color and \fuv{} luminosity, using the assumption that
the regions are SSPs. By comparing the ages to the SFHs, we found that most of
the regions are entirely inconsistent with the SSP assumption. Furthermore, the
ages often did not correspond to the main episodes of SF, and the masses were
discrepant with the masses integrated from the SFHs by up to 2 orders of
magnitude. These results call into question the practice of deriving ages and
masses for populations that are not confirmed SSPs.

We identified SSP-like regions as regions which formed 90\% or more of their
mass over the past $100\myr$ in a single age bin of their SFH. These
regions accounted for 18\% of our sample (6 of 33). Among this subset, we found
discrepancies of $10\myr$ in the ages and a factor of $3-4$ in the
masses derived from UV flux, most likely due to systematics in metallicity and
extinction. We propose that these discrepancies represent realistic
uncertainties in the SSP ages and masses, though the limited number of SSP-like
regions in our sample makes the uncertainties difficult to determine. Finally,
identification of the SSP-like regions was not possible from integrated \fuv{}
flux.

We have used SFHs to model the SEDs of over 9000 sub-\kpc{} regions in M31 and
produce detailed maps of synthetic UV flux across the entire PHAT survey area.
This work is an extensive follow-up to the analysis of \citet{Simones:2014},
which involved only 33 UV-bright regions from a small portion of the galaxy.
The SFHs were derived by \citet{Lewis:2014}\ using \acsb{} and \acsi{}
photometry from the PHAT survey. Both intrinsic and attenuated SEDs were
derived from the SFHs using FSPS. These were convolved with the GALEX \fuv{}
and \nuv{} response curves to obtain the synthetic intrinsic fluxes, \fxsfhz{},
as well as the synthetic attenuated fluxes, \fxsfh{}. All of the flux values
were then assembled into an overall map, or mosaic, using Montage. The mosaic
pixels corresponded to physical areas of $4.4\times 10^4\pc^2$. We constructed
corresponding maps for the observed flux, \fxobs{}, using GALEX DIS images.

The \fxsfh{} maps agreed with the \fxobs{} maps very well with respect to the
broad morphology of M31, faithfully reproducing all of the main features
brighter than $\sim 10^{-15}\uflambda$. We found the log ratios of \fxsfh{} to
\fxobs{} to be log-normally distributed with $\mu = 7.62\times 10^{-3}$ and
$\sigma = 2.37\times 10^{-1}$ for \fuv{}, and $\mu = -1.03\times 10^{-1}$ and
$\sigma = 1.59\times 10^{-1}$ for \nuv{}. In both filters, $\mu$ was within
$\sigma$ of a log flux ratio of 0, indicating that \fxsfh{} was consistent with
\fxobs{} on average. Due to the small pixel areas, the primary source of the
variance in the log flux ratios was most likely related to incomplete sampling
of the IMF.

We found no obvious trends in the flux ratios with respect to environment,
except for in the faintest, off-arm areas of the M31 where the variances in the
flux ratios were noticeably larger. We conclude that fluxes may be successfully
modeled from SFHs for any population in environments similar to M31. For our
sub-\kpc{} regions, we estimate the synthetic flux uncertainties to be
$10^\sigma = 1.7$ and 1.4 times the observed flux in \fuv{} and \nuv{},
respectively. Results from previous work on UV-bright regions by
\citet{Simones:2014} were consistent with our results.

The overall agreement between the observed and synthetic fluxes is remarkable
considering that our flux modeling procedure was dependent on several key
assumptions. Specifically, we assumed an IMF, models describing stellar spectra
and evolution, and an extinction model as well as an extinction curve. These
form the foundation for much research in astronomy and encompass our current
best understanding of stellar astrophysics and star formation. It is reassuring
that we can use all of this knowledge to successfully recreate detailed maps of
a galaxy from photometry in just two optical bands.

We used flux calibrations from \citet{Kennicutt:1998} with updates by
\citet{Hao:2011} and \citet{Murphy:2011} to estimate SFRs based on observed UV
flux, \sfrx{}. The \fxobs{} maps were first corrected for extinction using the
synthetic attenuated and intrinsic fluxes. We also calculated the $100\myr$
mean SFR from the SFHs, \sfroneh{}. We found that the faintest areas of M31 had
the highest ratios of \sfrx{} to \sfroneh{} and formed a linear tail feature in
plots of the SFR ratio versus \sfroneh{}. These tails were the result of a
distinct breakdown of the linear relationship between flux and SFR which
underpins the flux calibration method. We estimated a conservative threshold of
$\sfr \sim 10^{-5}\msun\yr^{-1}$ below which flux calibration should not be
used.

For the pixels above this threshold, we found the SFR ratios to be log-normally
distributed with $\mu = -2.46\times 10^{-1}$ and $\sigma = 2.61\times 10^{-1}$
for \fuv{}, and $\mu = 9.27\times 10^{-2}$ and $\sigma = 2.33\times 10^{-1}$
for \nuv{}. $\mu$ was within $\sigma$ of 0 for both the log SFR ratios in both
\fuv{} and \nuv{}, indicating that \sfrx{} was consistent with \sfroneh{} on
average. As for the flux ratios, incomplete sampling of the IMF was the main
source of the variance in the SFR ratios. We also considered deviations from
solar metallicity as well as SFH variability, and found that they were far less
important for the overall variances in the SFR ratios than IMF sampling.

Other than the faintest, off-arm areas which responsible for the tail feature
in the SFR ratio distributions, there were no found no obvious trends in the
SFR ratios with respect to environment. We determine that the flux calibration
method is safely applicable to environments similar to M31, \emph{but only as
long as the resulting \sfr{}s are greater than $\sim 10^{-5}\msun\yr^{-1}$}. We
estimate the SFR uncertainties for our sub-\kpc{} regions to be $10^\sigma =
1.8$ (\fuv{}) and 1.7 (\nuv{}) times the true, underlying $100\myr$-mean SFR.
The \sfrfuv{} uncertainty is slightly less than the factor of 2.5 uncertainty
previously estimated found by \citet{Simones:2014}.

We also measured global SFRs for the entire PHAT survey area. The global
\sfroneh{} value was $0.30\msun\yr^{-1}$, while the UV flux-based values were
$\sfrfuv = 0.22\msun\yr^{-1}$ and $\sfrnuv = 0.43\msun\yr^{-1}$. These values
are, respectively, about 30\% lower and 40\% higher than the global \sfroneh{},
and are well within the uncertainties derived from the SFR maps. The variances
in the SFR ratios due to IMF sampling is expected to decrease for larger areas,
however, so our estimated uncertainties should be considered firm upper limits
when applied to galaxies.

\textbf{*** TODO}

\begin{enumerate}
\item Compare and contrast the SFR results from Chapters \ref{uv_regions} and
    \ref{m31flux}. Are they consistent? How are they different and why?
\item Discuss the next steps for this line of work.
\end{enumerate}

\textbf{***}
