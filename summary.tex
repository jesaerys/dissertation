\chapter{Summary}
\label{summary}

In this study, we have derived the recent ($< 500\myr$) SFHs of 33
UV-bright regions in M31 using optical HST observations from PHAT. The regions
were defined by \citetalias{Kang:2009} based on GALEX \fuv{} surface brightness and
have areas ranging from $8 \times 10^3$ to $1.5 \times 10^6\pc^2$. We
used the SFH code MATCH to fit the CMDs of the regions and measure their the
SFHs based on the resolved stars from the PHAT photometry. We modeled the
extinction in the regions using a foreground parameter and a differential
parameter, which were optimized for each region to find the best-fit SFH.

We used FSPS to model both the intrinsic and reddened \fuv{} and \nuv{} magnitudes of
the regions based on their SFHs. The differences between the modeled reddened
and the observed \fuv{} magnitudes, $\fuvsfh - \fuvobs$, followed a
normal distribution with $\mu=0.09$ and $\sigma=0.3$. On average, the
\fuvsfh{} values were consistent with the \fuvobs{}
values, confirming the reliability of the SFHs, our extinction model, and the
\citet{Cardelli:1989} extinction curve. We attribute the scatter in the flux
ratios to the assumption made by FSPS that the IMF is fully populated while the
actual distribution of stellar masses becomes more discrete as smaller regions
are considered.

The observed, extinction-corrected \fuv{} magnitudes were converted into SFRs,
\sfrfuv{}, using the \fuv{} flux calibration from \citet{Kennicutt:1998}
with updates by \citet{Hao:2011} and \citet{Murphy:2011}. We also derived the mean
SFRs for the last $100\myr$ of the SFHs, \sfroneh{}. The $\sfrfuv / \sfroneh$
ratios were log-normally distributed with $\mu=0.2$
and $\sigma=0.4$. Overall, the \sfrfuv{} values were consistent with
the \sfroneh{} values, though a small amount of the
offset was attributable to inconsistencies with the metallicity assumed by the
flux calibration.

The intrinsic modeled \fuv{} magnitudes were also converted into SFRs,
\sfrfuvz{}, which were free from biases due to extinction
corrections and IMF sampling. The log-normal for the
$\sfrfuvz / \sfroneh$ ratios had $\mu=0.1$
and $\sigma=0.3$, indicating that assuming a constant SFR (implicit in the flux
calibration) for regions with highly variable SFHs is an important source of
scatter. We conclude that the total scatter in the $\sfrfuv / \sfroneh$ ratio
is due to the assumptions of a full IMF and a
constant SFR in regions where discrete sampling of the IMF and high variability
in the SFHs are important. Combined, these effects result in a factor of 2.5
uncertainty in the \fuv{}-based SFRs. Although there is a significant lack of
regions in our sample with areas between $10^5$ and $10^6\pc^2$, we
estimate that discrete IMF sampling and SFH variability become important below
$10^5\pc^2$, or scales of a few hundred pc.

Ages and masses were derived for the regions by \citetalias{Kang:2009} from
observed $\fuv - \nuv$ color and \fuv{} luminosity, using the assumption that
the regions are SSPs. By comparing the ages to the SFHs, we found that most of
the regions are entirely inconsistent with the SSP assumption. Furthermore, the
ages often did not correspond to the main episodes of SF, and the masses were
discrepant with the masses integrated from the SFHs by up to 2 orders of
magnitude. These results call into question the practice of deriving ages and
masses for populations that are not confirmed SSPs.

We identified SSP-like regions as regions which formed 90\% or more of their
mass over the past $100\myr$ in a single age bin of their SFH. These
regions accounted for 18\% of our sample (6 of 33). Among this subset, we found
discrepancies of $10\myr$ in the ages and a factor of $3-4$ in the
masses derived from UV flux, most likely due to systematics in metallicity and
extinction. We propose that these discrepancies represent realistic
uncertainties in the SSP ages and masses, though the limited number of SSP-like
regions in our sample makes the uncertainties difficult to determine. Finally,
identification of the SSP-like regions was not possible from integrated \fuv{}
flux.


\textbf{*** TODO}

\begin{enumerate}
\item Insert the summary from Chapter \ref{m31flux} here.
\item Compare and contrast the SFR results from Chapters \ref{uv_regions} and
    \ref{m31flux}. Are they consistent? How are they different and why?
\item Discuss the next steps for this line of work.
\end{enumerate}

\textbf{***}
