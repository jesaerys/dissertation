I thank my adviser, Evan Skillman, for all of his time and effort spent guiding
me to my PhD. He has provided me with many wonderful opportunities that have
truly enriched my graduate school experience. I have met great people, traveled
to exciting places, and worked on interesting projects all because of him, and
I cannot imagine how my trajectory would be different had he not accepted me as
a student so many years ago. He always has great advice, both academic and on
life. Without the latter, this dissertation may never have been written. Above
all, Evan is a patient and understanding person, and it has been a pleasure
working for him. Plus, he's a riot at dinner.

I must also thank my graduate comrades, both those who are sill toiling away
and those who have moved on to better things. Graduate school is not for the
faint of heart, but good laughs and lots of beer make it easier. The friends I
have made over the years have been vital for balancing the bad with the good,
and have made this whole experience worth something more than just attaining a
degree.

For my parents, words simply cannot express my love and gratitude. While I was
growing up, Mom and Dad both taught me the importance of education and I always
knew they expected me to go to college. However, I never felt pressured to
pursue any given profession. They allowed me to discover myself on my own terms
and actively fostered my interests, even when my interests were borderline
ridiculous. (Remember when I thought I wanted to be a rockstar?) As for most
young adults, life got hard when I started college, and even more so when I
started graduate school. I quickly discovered my limits, that there is only so
much I can do on my own, and my parents have helped me out on countless
occasions along the way, allowing me to focus on my studies. The only reason I
have made it this far is with their constant, unconditional support.

I give my most heartfelt thanks to my wife, Annie, who, for better or worse,
has stuck with me on this trying journey. I remember when we both started our
professional studies. We were mostly care-free and could not have been happier
chasing our ultimate academic goals -- and we had no idea what challenges were
ahead of us. The hours slowly grew longer, the dinners less frequent, and the
stress became heavy. I even got my first gray hairs. Yet Annie has been there
for me through all of it. She always made me feel better after a bad exam and
shared my joy in my successes, and she is still the person I most look forward
to seeing at the end of the day (and I miss her when I don't). She always
inspired me to keep going and provided words of encouragement when I needed it
-- except when she didn't. After all, she has never been afraid to force me to
think about what I really want in life and sometimes that meant having
difficult conversations. Deep down she is pragmatic and I like her for that,
too, because she makes me a better person. We have been through so much
together and finally earning my degree is as much her victory as it is mine.
She is my equal, my best friend, and the love of my life.
