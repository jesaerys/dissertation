\chapter{Introduction}
\label{introduction}

\section{Stellar population modeling}

SSP modeling requires stellar evolution models, stellar spectral libraries, and
IMF; all of these (except the IMF, if it's universal) depend on metallicity.

SPS codes are typically based on one dimensional models, but processes like
convection, rotation, mass-loss, close binary interactions, and thermal
pulsation are inherently three dimensional.

Core and envelope ``overshooting'' refers to in increase in size of a
convective layer relative to a more classical radiative-convective boundary.
The amount of overshooting is essentially a free parameter that is tuned by
observations.

SPS codes typically do not include binary evolution.

Input models are not usually completely theoretical, but rather
``semi-empirical'' such that the physics of the models are tuned using
observations. For example, the TP-AGB component of the Padova isochrones
\citep{Marigo:2008} was calibrated using observations of nearby galaxies
\citep{Girardi:2010}.




\section{My introduction 1}

A common technique for estimating global star formation rates (SFRs) in
individual galaxies is to measure the total flux at wavelengths known to trace
recent star formation (SF), such as ultraviolet (UV) emission from
intermediate- and high-mass stars. After correcting for dust extinction, an
observed flux can be converted into a SFR using a suitable calibration, which
is typically a linear scaling of intrinsic luminosity derived from population
synthesis modeling. The modeling process requires a set of stellar evolution
models and a stellar initial mass function (IMF), as well as a characterization
of the star formation history (SFH; the evolution of SFR over time) and the
metallicity of the population. These quantities are often not well-constrained
for a given system and need to be assumed (see reviews by
\citealt{Kennicutt:1998}, \citealt{Kennicutt:2012}, and references therein).

A set of flux calibrations widely used in extragalactic studies were presented
by \citet[][see \citealp{Kennicutt:2012} for updates]{Kennicutt:1998}. These
calibrations are based on models of a generic population with solar
metallicity, a fully populated IMF, and a SFR that has been constant over the
lifetime of the tracer emission ($\sim 100\myr$ for UV). The flux calibrations
are therefore applicable to any population that can be assumed to approximate
the generic population, such as spiral galaxies. In environments with low total
SF (i.e., low mass) or on subgalactic scales, however, the assumptions of a
fully populated IMF and a constant SFR start to become tenuous. As a result,
applying the flux calibrations in these situations can lead to inaccurate SFR
estimates.

For populations located within a few Mpc, it is possible to measure SFRs more
directly by fitting the color magnitude diagram (CMD) of the resolved stars to
obtain a SFH \citep{Dolphin:2002}. At its core, CMD fitting is a population
synthesis technique just like flux calibration (albeit much more complex) and
thus requires a set of stellar evolution models, an IMF, and an accounting of
dust. The primary advantage of CMD fitting over the flux calibration method for
obtaining SFRs, however, is the elimination of assumptions about the SFH and
metallicity. CMD-based SFHs thus provide a relative standard for testing the
accuracy of SFR estimates from commonly used flux calibrations, especially in
applications where the underlying full-IMF and constant-SFR assumptions are not
strictly satisfied. More generally, the SFHs can be used to test results from
any other flux-based method, such as ages and masses derived under the simple
stellar population (SSP) assumption.

With recent Hubble Space Telescope (HST) observations from the Panchromatic
Hubble Andromeda Treasury \citep[PHAT;][]{Dalcanton:2012}, we have measured the
recent SFHs ($< 500\myr$) of 33 UV-bright regions in M31 and compared
them with SFRs derived from UV flux. We also compared the SFHs with ages and
masses derived from UV flux by treating the regions as SSPs. The UV-bright
regions were cataloged by \citet[][\citetalias{Kang:2009} hereafter]{Kang:2009} using
Galaxy Evolution Explorer (GALEX) far-UV (\fuv{}, $\lambda \sim
1540\ang$) flux and have areas ranging from $10^4$ to
$10^6\pc^2$. This range of sizes allowed us to test the reliability
of the full-IMF, constant-SFR, and SSP assumptions on sub-kpc scales.

This paper is organized as follows. We describe our sample of UV-bright regions
and show their CMDs from the PHAT photometry in \S \ref{uvr:observations}. We
summarize the CMD-fitting process, describe our extinction model, and present
the resulting SFHs of the regions in \S \ref{uvr:sfhs}. \S \ref{uvr:fluxmod} describes
the modeling of UV magnitudes from the SFHs, and \S \ref{uvr:sfrs} describes the
total masses and the mean SFRs from the SFHs, as well as the SFRs based on UV
flux. In \S \ref{uvr:discussion}, we compare the UV flux-based SFRs, ages, and
masses with the results from the SFHs, discuss the applicability of the
full-IMF, constant-SFR, and SSP assumptions to our sample, and attempt to
quantify the uncertainties associated with using UV flux to estimate SFRs,
ages, and masses for sub-kpc UV-bright regions.




\section{My introduction 2}

M31 is well-studied, $\sim L_\ast$ galaxy and has been observed at a variety of
wavelengths, e.g., in the ultraviolet (UV) by the Galaxy Evolution Explorer
\citep[GALEX;][]{Morrissey:2007}, in the optical, including H$\alpha$, for the
Local Group Galaxies Survey \citep{Massey:2006}, and in the infrared by the
Spitzer Space Telescope \citep{Gordon:2006}. The wealth of high-quality data
available for M31 provides a valuable opportunity to model various observations
and test our current understanding of stellar astrophysics. In particular, the
initial mass function (IMF), stellar evolution and spectra models, and
extinction curves are all required to model the light produced by a galaxy.

A critical ingredient for modeling the flux from a galaxy is a detailed
knowledge about its underlying star formation history (SFH). Deriving SFHs by
color-magnitude diagram (CMD) analysis is a reliable method that can be used
whenever photometry of resolved stars is available. An extensive optical
photometric catalog for M31 has been compiled by the Panchromatic Hubble
Andromeda Treasury \citep[PHAT][]{Dalcanton:2012}, and \citet{Lewis:2014} have
used these data to derive the spatially-resolved SFH of the northeast quadrant.
With sub-\kpc{} resolution, this SFH dataset is the ideal input for stellar
population synthesis codes that model total flux given a population's star
formation rate (SFR) and metallicity evolution. The result is a set of
spatially-resolved maps of synthetic broadband flux in M31 which can be
compared with observations.

The \citet{Lewis:2014} SFHs can also be used to create temporally-average SFR
maps. Because the SFHs were derived from the resolved stars without any prior
assumptions about the SFHs, such maps provide a standard against which
flux-based SFR estimates \citep[e.g., using any of the calibrations
from][]{Kennicutt:2012} can be tested. Using integrated flux to estimate SFRs
for distant galaxies where resolved stars are not available is a common
technique in extragalactic astronomy. Some studies have investigated how
flux-based SFR estimators hold up against resolved-star SFHs in sub-\kpc{}
UV-bright regions \citep{Simones:2014} and in low-metallicity dwarf galaxies
\citep{McQuinn:2014}, yet relatively little attention has been paid to the
accuracy of such estimators in general.

In this study, we have used the PHAT CMD-based SFHs and stellar population
synthesis to create maps of synthetic ultraviolet (UV) flux at sub-\kpc{}
resolution for the northeast quadrant of M31. We then compared the synthetic
flux maps with observations from GALEX. We currently only focus on GALEX \fuv{}
and \nuv{} (far and near UV), though this work can easily be extended to other
wavelength regimes. In \S \ref{mfx:syntheticfluxmaps}, we describe the SFH
dataset and the production of the synthetic flux maps. \S
\ref{mfx:observations} describes the process of producing observed flux maps
from GALEX \fuv{} and \nuv{} images. The creation of SFR maps both from the
SFHs and the observed fluxes using common flux-SFR calibrations are described
in \S \ref{mfx:sfrestimates}. In \S \ref{mfx:discussion}, we compare the
synthetic maps with the observations and compare mean SFR maps with SFRs
estimated from observed flux. We conclude in \S \ref{mfx:conclusion}.


